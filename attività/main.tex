\documentclass[a4paper]{article}
\usepackage{listings}
\usepackage{xcolor}
\usepackage{forest}
\usepackage{changepage}
\usepackage{hyperref}
\usepackage{listings}
\usepackage[utf8]{inputenc}
\usepackage[italian]{babel}
\usepackage{csquotes}
\usepackage{enumitem}
\graphicspath{{figures/}}
\usepackage{xcolor}

\begin{document}

\author{Lorenzo Dentis, lorenzo.dentis@edu.unito.it}
\title{Consegna Finale}
\maketitle
\section{appunti}
rivedere la lez del 24.11 e del 28.11 (parte 1 e 2)
Esercizio unplugged: analisi delle frequenze di un testo sarebbe figo.

Esercizio al pc. cifrario a sostituzione, analizza il programma che genera il cifrato (con ausilio di debugger) e prova a capire come decifrare un messaggio citato con tale algoritmo. (PRIMM, predict e run)
Si potrebbe anche pensare di scrivere il codice di un programma che decifri. 
\end{document}
