\documentclass[a4paper]{article}
\setcounter{secnumdepth}{3}
\usepackage{listings}
\usepackage{qtree}
\usepackage{xcolor}
\usepackage{forest}
\usepackage{multicol}
\setlength{\columnsep}{3cm}
\usepackage{parskip}
\usepackage{changepage}
\usepackage[T1]{fontenc}
\usepackage{amsmath}
\usepackage{hyperref}
\usepackage{listings}
\usepackage{amsthm}
\usepackage{amssymb}
\usepackage{float}
\usepackage[utf8]{inputenc}
\usepackage{graphicx}
\usepackage[italian]{babel}
\usepackage{thmtools}
\usepackage{csquotes}
\usepackage{biblatex}
\graphicspath{{figures/}}
\usepackage{xcolor}
\addbibresource{refs.bib}

\begin{document}

\author{Lorenzo Dentis, lorenzo.dentis@edu.unito.it}
\title{Consegna 6 \\ \large la natura dei programmi}
\maketitle

\section{Domande}
Partendo dal materiale relativo al seminario tenuto dalla Dott.ssa Violetta Lonati Di cosa parliamo quando parliamo di programmi scrivete una breve riflessione su questi aspetti:
\begin{itemize}
	\item Nei corsi di studio che avete affrontato sono emerse tutte le sfaccettature del concetto di programma presentate nel seminario?
	\item Guardando le indicazioni nazionali del laboratorio Informatica e Scuola del cini e quelle degli istituti superiori vi sembra che includano tutte le 6 facce dei programmi?
	\item Dovendo progettare un intero corso di scienze informatiche in una scuola superiore in che ordine presentereste le 6 sfaccettature?
\end{itemize}
\section{Risposte}
\subsection{1 Esperienze personali}
Io ho svolto il liceo scientifico delle scienze applicate ed ho avuto la fortuna di avere un buon professore di informatica.\\
Ho affrontato quasi tutti i sei campi, ma principalmente concentrandoci sui programmi come oggetti nozionali (abbiamo approfondito molto il linguaggio C ad esempio) e programmi come oggetti astratti (parecchia enfasi sulla differenza tra algoritmo e codice, parecchio tempo dedicato al problem solving piuttosto che alla scrittura di codice)
C'è stata molta enfasi anche riguardo al tema "programmi come entità eseguibili".




Differentemente da ciò che si immagina non abbiamo visto molto il programma come strumento, che io ricordi c'è stato poco focus sulle competenze digitali o sull'informatica in ambito interdisciplinare. A parte naturalmente l'utilizzo di strumenti a supporto dello sviluppo di algoritmi o della scrittura di codice (Scratch, Algobuild, IDE vari, ....)
Un'altra sfaccettatura che non è stata quasi affrontata è stata la visione del programma come opera dell’uomo, a parte una lezione 

\end{document}
