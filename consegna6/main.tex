\documentclass[a4paper]{article}
\setcounter{secnumdepth}{3}
\usepackage{listings}
\usepackage{xcolor}
\usepackage{forest}
\usepackage{multicol}
\setlength{\columnsep}{3cm}
\usepackage{parskip}
\usepackage{changepage}
\usepackage[T1]{fontenc}
\usepackage{amsmath}
\usepackage{hyperref}
\usepackage{listings}
\usepackage{amsthm}
\usepackage{amssymb}
\usepackage{float}
\usepackage[utf8]{inputenc}
\usepackage{graphicx}
\usepackage[italian]{babel}
\usepackage{thmtools}
\usepackage{csquotes}
\usepackage{biblatex}
\graphicspath{{figures/}}
\usepackage{xcolor}
\addbibresource{refs.bib}

\begin{document}

\author{Lorenzo Dentis, lorenzo.dentis@edu.unito.it}
\title{Consegna 6 \\ \large la natura dei programmi}
\maketitle

\section{Domande}
Partendo dal materiale relativo al seminario tenuto dalla Dott.ssa Violetta Lonati Di cosa parliamo quando parliamo di programmi scrivete una breve riflessione su questi aspetti:
\begin{itemize}
	\item Nei corsi di studio che avete affrontato sono emerse tutte le sfaccettature del concetto di programma presentate nel seminario?
	\item Guardando le indicazioni nazionali del laboratorio Informatica e Scuola del cini e quelle degli istituti superiori vi sembra che includano tutte le 6 facce dei programmi?
	\item Dovendo progettare un intero corso di scienze informatiche in una scuola superiore in che ordine presentereste le 6 sfaccettature?
\end{itemize}
\section{Risposte}
\subsection{1 Esperienze personali}
Ho frequentato il liceo scientifico delle scienze applicate e ho avuto la fortuna di avere un buon insegnante di informatica. 
Abbiamo coperto quasi tutti i sei campi, ma ci siamo concentrati principalmente sui programmi come oggetti nozionali, approfondendo molto il linguaggio C ad esempio, e sui programmi come oggetti astratti, con molta enfasi sulla differenza tra algoritmo e codice e sul problem solving piuttosto che sulla scrittura di codice.
È stato dato molto spazio anche al tema dei programmi come entità eseguibili, poiché abbiamo fatto molta programmazione.


Contrariamente a quanto si potrebbe pensare, non abbiamo visto molto il programma come strumento, ricordo pochi riferimenti alle competenze digitali o all'informatica in ambito interdisciplinare. Tuttavia, abbiamo utilizzato strumenti a supporto dello sviluppo di algoritmi o della scrittura di codice, come Scratch, Algobuild e vari IDE.\\
Non è stata quasi affrontata la visione del programma come opera dell'uomo, ad eccezione di una lezione in cui sono stati affrontati aspetti del ruolo dell'informatica nella storia, parlando di Alan Turing e della macchina Colossus. 
Non è stato affrontato il tema del programma come oggetto fisico, nonostante una breve introduzione all'hardware, all'architettura e alle reti, ma mai in relazione ai programmi.
Il professore ha però proposto una attività facoltativa durante una autogestione riguardo alla programmazione di microcontrollori (Arduino)


In ambito universitario sono state ampiamente approfondite tutte e sei le sfaccettature.
\end{document}
