\documentclass[a4paper]{article}
\setcounter{secnumdepth}{3}
\usepackage{listings}
\usepackage{qtree}
\usepackage{xcolor}
\usepackage{forest}
\usepackage{multicol}
\setlength{\columnsep}{3cm}
\usepackage{parskip}
\usepackage{changepage}
\usepackage[T1]{fontenc}
\usepackage{amsmath}
\usepackage{hyperref}
\usepackage{listings}
\usepackage{amsthm}
\usepackage{amssymb}
\usepackage{float}
\usepackage[utf8]{inputenc}
\usepackage{graphicx}
\usepackage[italian]{babel}
\usepackage{thmtools}
\usepackage{csquotes}
\usepackage{biblatex}
\graphicspath{{figures/}}
\usepackage{xcolor}
\addbibresource{refs.bib}

\begin{document}

\author{Lorenzo Dentis, lorenzo.dentis@edu.unito.it}
\title{Consegna 1 \\ \large Domande parte introduttiva}
\maketitle

\section{Domande}
Rispondere alle seguenti domande. Le risposte devono essere motivate citando le letture indicate su moodle. Dovete esprimere la vostra opinione  rispetto alle posizioni espresse dagli articoli degli autori: siete d’accordo? se si/no perchè?
\newline
\begin{enumerate}
	\item E’ importante insegnare informatica come materia scolastica?
	\item L’informatica è una scienza?
	\item Qui sotto trovate una lista di criteri estratti dall’articolo The Science in Computer Science. Sono usati dall’autore per definire la credibilità di un settore come “scienza”. Siete d’accordo con la scelta di questi criteri? Motivare le risposte:
Organized to understand, exploit, and cope with a pervasive phenomenon.
\begin{itemize}
	\item Organized to understand, exploit, and cope with a pervasive phenomenon
	\item Encompasses natural and artificial processes of the phenomenon.
	\item Codified structured body of knowledge.
	\item Commitment to experimental methods for discovery and validation.
	\item Reproducibility of results.
	\item Falsifiability of hypotheses and models.
	\item Ability to make reliable predictions, some of which are surprising.
\end{itemize}
	L’informatica soddisfa qualcuno dei criteri sopra elencati? Se si quali? e perchè?
\end{enumerate}
\newpage
\section{Risposte}
\subsection{È importante insegnare informatica come materia scolastica?}
Secondo me insegnare l'informatica nelle scuole è importante, ma ancor più importante è insegnare le competenze digitali.\\
Ogni cittadino dovrebbe essere in grado di operare un computer o device digitale, perchè ci stiamo muovendo verso una società che per svoglere anche compiti necessare alla basilare sopravvivenza richiede di utilizzare dispositivi informatici. Saper utilizzare correttamente un computer e navigare su internet sta diventando una competenza basilare tanto quanto leggere e scrivere.\\
\\
Parlando invece di \emph{Informatica} invece credo che fornire delle basi sia importante.\\
Come l'autore di "Informatica e competenze digitali: cosa insegnare? "\cite{insegnare} , anche io credo che la necessità dell'insegnamento di questa disciplina derivi dal voler \emph{"preparare i cittadini a comprendere appieno la società digitale"}.
Trovo che l'Informatica come materia scolastica abbia molto in comune con la Storia, in quanto, anche nel caso in cui non rilevante nella crescita accademica di un individuo, è fondamentale per capire il mondo intorno a noi. Il manifesto per l'umanesimo digitale\cite{umanesimo} parla molto di come le teconologie dovrebbero essere adattate a supporto della libertà e della democrazia, ma secondo me deve esserci anche una modifica del comportamento umano nei confronti della tecnologia.\\
Non si può avere democrazia senza conoscenza. Anche presuppondendo di avere un prodotto perfetto e costruito con le migliori intenzioni, che rispetta la privacy dell'utente e garantisce imparzialità, tale strumento è inutile se il suo utilizzatore non si fida e non ci può essere fiducia se prima lo strumento non viene compreso.
Un buon esempio può essere il voto elettronico. Se anche venisse realizzata una piattaforma online perfetta poi bisognerebbe convincere le persone ad usarla.Io stesso, come immagino chiunque altro, mi rifiuterei di votare se questa fosse una "black box" dal funzionamento sconosciuto.

Quindi concordo con l'affermazione \emph{"L‘educazione all‘informatica e al suo impatto sociale devono iniziare il prima possibile"}, tratta dal manifesto sopra citato,perchè bisogna fornire ai futuri cittadini le competenze per utilizzare e sopratutto comprendere il funzionamento dei nuovi strumenti tecnologici che permeano tutti gli aspetti della nostra vita.

\subsection{L’informatica è una scienza?}
\label{sec:Q2}
Si, come affermato durante le lezioni ciò dipende molto da cosa si considera scienza e cosa no.
Io mi rifaccio alla definizione data da Galileo Galilei, cioè che una disciplina è scienza quando rispetta il metodo scientifico.
Quindi una disciplina che segue le seguenti fasi: osservazione di un fenomeno, misura del fenomeno, formulazione di una ipotesi, preparazione di un esperimento che provi l'ipotesi e che sia ripetibile. Nel caso dell'informatica ovviamente l'oggetto di studio è una informazione.\\
In particolare io "traccio la linea di demarcazione" tra scienza e non scienza sulla ripetibilità di un esperimento. Ad esempio non considero scienza le scienze sociali o l'economia, perchè gli esperimenti non possono essere effettuati "in ambiente sterile" ed i risultati ottenuti da differenti esperimenti non saranno mai identici, perchè è impossibile partire dalle stesse condizioni di partenza.\\
In informatica è invece partendo dalle stesse condizioni iniziali (che sono completamente controllabili) ed eseguendo lo stesso algoritmo si giunge sempre alle stesse conclusioni, l' informatica è deterministica.

\subsection{Siete d’accordo con la scelta di questi criteri?}
\emph{Organized to understand, exploit, and cope with a pervasive phenomenon.}\\
Lo scopo della scienza è senza dubbio quello di dare una spiegazione ai fenomeni che avvengono intorno a noi.Gli altri 2 criteri "exploit and cope" trovo non appartengano tanto ad una disciplina scientifica quanto ad una disciplina \emph{Tecnica}.Secondo me, basterebbe il primo criterio, senza includere gli altri due, d'altra parte questi sono conseguenza del primo.Una volta compreso un fenomeno è naturale che spontaneamente sorgano metodi per sfruttare questa conoscenza, altrimenti non ci si sarebbe interessati al fenomeno in primo luogo.
\\\\

\emph{Encompasses natural and artificial processes of the phenomenon}.\\
Anche riguardo a questo criterio non sono convinto, trovo sia irrilevante la natura del fenomeno.Non vedo perchè una disciplina che studi fenomeni artificiali non dovrebbe essere considerata scienza.
\\\\

\emph{Codified structured body of knowledge}.\\
Si, credo che ogni scienza per essere considerata tale abbia bisogno di basi, siano questi gli assiomi ed i teoremi della matematica o le varie teorie della biologia.
Come poi è strutturato questo insieme di informazioni dipende da una scienza all'altra, ma senza una struttura diventa "una storia", letteratura piuttosto che scienza.
\\\\

\emph{Commitment to experimental methods for discovery and validation}.\\
Questo secondo me è il principale distinguo tra scienza e non scienza.Il metodo scientifico sviluppato da Galileo e la seguente elaborazione di Bacon sono a mio parere il punto di partenza per ogni scienza.Ad una ipotesi deve sempre seguire una validazione logica e sperimentale, altrimenti non si può considerare provata una teoria.
\\\\

\emph{Reproducibility of results}.\\
Tale criterio segue dal criterio precedente, un esperimento per essere considerato rilevante deve essere ripetibile, come scritto nella domanda \ref{sec:Q2}, non credo che le discipline che non soddisfano tale requisito siano definibili \emph{scienza}
\\\\

\emph{Falsifiability of hypotheses and models}.\\
La scienza deve avere la possibilità di evolversi con il susseguirsi di nuove scoperte, non c'è posto nella scienza per concetti immutabili o "atti di fede".Anche le teorie più solide devono essere falsificabili.Come nel caso del famoso "paradosso dei corvi" \cite{corvi}, basta una controprova per rendere falsa una teoria.
\\\\

\emph{Ability to make reliable predictions, some of which are surprising.}\\ Mi interrogo sul perchè una scienza debba per forza ottenere risultati sorprendenti, dato che questo è un metro di valutazione soggettivo.Invece riguardo all'abilita di fare predizioni sono abbastanza convinto.Come accennato nel primo punto è una naturale conseguenza della comprensione di un fenomeno. 

\subsection{Quali criteri soddisfa l'informatica?}
\emph{Organized to understand, exploit, and cope with a pervasive phenomenon.}\\
Il fenomeno pervasivo che l'informatica studia è l'informazione e l'elaborazione della stessa.Non credo esista nulla di più pervasivo delle informazioni. Ogni fenomeno porta con se informazioni, dal comportamento umano alla biologia alla chimica.I metodi con cui noi umani abbiamo imparato a sfruttare la raccolta e l'elaborazione di informazioni sono abbastanza palesi, il primo che mi viene in mente è Internet.
\\\\

\emph{Encompasses natural and artificial processes of the phenomenon}.\\
Come dice il documento di Peter J. Denning\cite{isScience} l'arogmentazione "l'informatica studia solo processi artificiali" cade quando si considera il fatto che l'informatica è la disciplina che studia le informazioni, non le tecnologie. 
L'articolo \cite{isScience}invita anche a dibattere sul fatto che tutte le informazioni trattate dall'informatica derivano da fenomeni artificiali, anche le previsioni del tempo di un algoritmo metereologico non sono lo studio dell'evento metereologico in se, ma lo studio della simulazione dello stesso. Ciò non toglie che "le informazioni" non sono per forza artificiali, anche una reazione chimica porta con se delle informazioni, quindi io credo che l'informatica soddisfi questo criterio, seppur non strettamente.\\ \\

\emph{Codified structured body of knowledge}.\\La conoscenza dell'informatica è basata su principi teorici, modelli matematici e algoritmi che sono stati sviluppati e codificati nel corso degli anni. Basti pensare a tutti gli studi di informatica teorica quali algoritmi, complessità e calcolabilità, ma io includerei anche argomenti come i sistemi operativi, le reti, la sicurezza informatica, l'analisi dei dati, l'intelligenza artificiale e molti altri. \\\\

\emph{Commitment to experimental methods for discovery and validation}.\\Questo è un altro punto su cui si può dibattere, in quanto l'informatica di uso quotidiano non fa uso di esperimenti.Quando viene scritto un software difficilmente viene prima validato (nonostante ci siano strumenti per farlo).D'altra parte si potrebbe considerare l'uso da parte dell'utente come una sperimentazione, in quanto il software viene costantemente aggiornato, una specie di sperimentazione sul campo.\\In ogni caso questa argomentazione decade quando si pensa a software sviluppato per scopo di ricerca.Mi viene in mente l'esempio del Q-learning, algoritmo di apprendimento automatico che si basa proprio sull'idea di effettuare molte azioni differenti e trovare quella che fornisce un risultato migliore in fase di addestramento, cioè esattamente sperimentare differenti strategie finchè nion si trova quella migliore.\\\\

\emph{Reproducibility of results}.\\Ho ampiamente discusso di questo punto nella domanda precendente \ref{sec:Q2}.Partendo da un ambiente controllato, situazione molto semplice da produrre in informatica dato che siamo noi a generare gli input, gli algoritmi utilizzati produrranno output prevedibili.\\\\

\emph{Falsifiability of hypotheses and models}.\\Come ho accenanto in precedenza ci sono metodi formali per la validazione del software, che poi questi non vengano usati è un altro discorso.Il documento di Peter J.Denning fa lo stesso ragionamento (portando a supporto di questa tesi il fatto che circa il 50\% dei modelli e delle ipotesi proposte non erano state rigorosamente testate).Direi che l'informatica soddisfa questo requisito se non fosse per alcuni problemi indecidibili, primo tra tutti l'\emph{halting problem} di Turing.In informatica (come in matematica) ci sono dei problemi indecidibili, di conseguenza non tutte le ipotesi ed i modelli sono verificabili falsi.Quindi a mio parere l'informatica non soddisfa questo criterio.\\\\

\emph{Ability to make reliable predictions, some of which are surprising.}\\Riguardo al fatto che l'informatica permetta di effettuare predizioni ho già speso molte parole.Sul fatto che siano sorprendeti mi sento di affermare che senza dubbio l'informatica permette di fare scoperte quantomeno non intuitive.
I risultati delle scoperte informatiche hanno radicalmente cambiato il mondo forse più di quanto qualsiasi altra scienza abbia fatto.Un esempio di predizione sorprendente a mio parere è la teoria della complessità. Un problema semplice come le torri di Hanoi quando viene esteso sopra una certa soglia richiede un tempo computazionale sorprendente, cosa che non ci si aspetterebbe.

\printbibliography
\end{document}
