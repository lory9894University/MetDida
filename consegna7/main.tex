\documentclass[a4paper]{article}
\setcounter{secnumdepth}{3}
\usepackage{listings}
\usepackage{qtree}
\usepackage{xcolor}
\usepackage{forest}
\usepackage{multicol}
\setlength{\columnsep}{3cm}
\usepackage{parskip}
\usepackage{changepage}
\usepackage[T1]{fontenc}
\usepackage{amsmath}
\usepackage{hyperref}
\usepackage{listings}
\usepackage{amsthm}
\usepackage{amssymb}
\usepackage{float}
\usepackage[utf8]{inputenc}
\usepackage{graphicx}
\usepackage[italian]{babel}
\usepackage{thmtools}
\usepackage{csquotes}
\usepackage{biblatex}
\graphicspath{{figures/}}
\usepackage{xcolor}
\addbibresource{refs.bib}

\begin{document}

\author{Lorenzo Dentis, lorenzo.dentis@edu.unito.it}
\title{Consegna 3 \\ \large Algoritmo}
\maketitle

\section{Domande}
Tenendo conto delle discussioni svolte a lezione sulla definizione di algoritmo e  delle sue proprietà scrivete una vostra definizione di algoritmo e delle sue proprietà motivando eventuali differenze rispetto alla formulazione vista a lezione. Nello svolgere  questo compito tenete presente che la discussione sulle definizioni deve essere rivolta a classi delle scuole secondarie di secondo grado senza prerequisiti particolari.
\section{Risposte}
\subsection{Necessità di un costrutto di programmazione}
Mi è capitato due volte che io ricordi. Un esempio "voluto" si presenta nel corso "Algoritmi e Complessità", ove il professor Roversi presenta 3 problemi irrisolvibili con la nostre conoscenze attuali con lo scopo di mostrarci questa nostra "carenza". Un esempio di "Productive failure" (metodologia Costruttivista che abbiamo visto nel secondo argomento del corso).\\ 
Invece in maniera più "naturale" mi è capitato con i Design Pattern, ricordo di aver avuto necessità di una programmazione più "strutturata". Ê successo molto tempo fa ma se non vado errato mi era capitato di aver necessità di utilizzare un Singleton (al tempo senza sapere cosa fosse) e di aver pensato "Ci sarà un modo di utilizzare una classe come se fosse un oggetto", aver cercato a lungo su internet ed aver scoperto il Singleton Pattern.

\end{document}
