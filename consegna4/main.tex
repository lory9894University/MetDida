\documentclass[a4paper]{article}
\setcounter{secnumdepth}{3}
\usepackage{listings}
\usepackage{qtree}
\usepackage{xcolor}
\usepackage{forest}
\usepackage{multicol}
\setlength{\columnsep}{3cm}
\usepackage{parskip}
\usepackage{changepage}
\usepackage[T1]{fontenc}
\usepackage{amsmath}
\usepackage{hyperref}
\usepackage{listings}
\usepackage{amsthm}
\usepackage{amssymb}
\usepackage{float}
\usepackage[utf8]{inputenc}
\usepackage{graphicx}
\usepackage[italian]{babel}
\usepackage{thmtools}
\usepackage{csquotes}
\usepackage{biblatex}
\graphicspath{{figures/}}
\usepackage{xcolor}
\addbibresource{refs.bib}

\begin{document}

\author{Lorenzo Dentis, lorenzo.dentis@edu.unito.it}
\title{Consegna 4 \\ \large progettazione soluzione/algoritmo}
\maketitle

\section{Domande}
\begin{itemize}
	\item Leggete l'articolo Programming Patterns and Design Patterns in the Introductory Computer Science Course e scrivete un breve resoconto sull'approccio proposto:siete d'accordo con l'approccio proposto? lo adottereste? immaginate di proporlo in una classe in cui insegnate: quali sono le potenziali difficoltà?
	\itemi riportate il testo del problema analizzato per l'attività 2) Progettazione soluzione/algoritmo e la soluzione che avete progettato: suddivisione in sottoproblemi,pattern algoritmici, elementari e ruoli delle variabili.
	\item descrivete anche eventuali critiche (es. sulla formulazione del problema), problemi e considerazioni emerse durante l'attività
\end{itemize}
\newline
\section{Risposte}
\subsection{1. Commento sull'articolo}
In generale mi trovo molto d'accordo con il metodo presentato nell'articolo \cite{articolo}, anche perchè è un metodo molto simile a come mi trovo bene ad imparare nuovi argomenti.
Come scritto in sezione 4.2 subito dopo \textit{Intent and Motivation} viene la sezione \textit{Problem Examples} in cui vengono mostrati un paio di esempi di utilizzo del Pattern.\\
Trovo che, dovendo affrontare un nuovo problema, sia molto d'aiuto vedere subito un esempio di soluzione in atto.\\
In aula una collega sosteneva che questo metodo potrebbe essere di intralcio all'apprendimento del \textit{problem solving}.
In effetti cercare di categorizzare ogni problema e fornire la soluzione agli studenti rende la soluzione molto "meccanica" e non aiuta a sviluppare la capacità di "cercare" un modo di risolvere il problema.
Gli autori cercano di semplificare il problema del "I do not even know where to start" ma molto spesso capire da dove partire per risolvere un problema è una skill di per se, una abiità che bisogna sviluppare 

\end{document}
